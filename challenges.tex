%%%%%%%%%%%%%%%%%%%%%%%%%%%%%%%%%%%%%%%%%%%%%%%%%%%%%%%%%%%%%%%%%%%%%%%%%%%
%% Rulebook RoboCup Logisitc League
%% 2021 Challenges
%%%%%%%%%%%%%%%%%%%%%%%%%%%%%%%%%%%%%%%%%%%%%%%%%%%%%%%%%%%%%%%%%%%%%%%%%%%
% chktex-file 29
\documentclass[12pt,twoside]{article}
\usepackage[a4paper]{anysize}
\marginsize{2cm}{2cm}{1cm}{2cm}

\setlength{\marginparwidth}{1cm}
%\usepackage{fourier}

\usepackage{floatpag}
\usepackage{wrapfig}
\usepackage{fnpct}

\usepackage[table]{xcolor}

\usepackage[binary-units=true]{siunitx}
%% HYPERREF %%%%%%%%%%%%%%%%%%%
\usepackage{hyperref}
\hypersetup{
  pdftitle      = {The RoboCup Logistics League Challenge Rulebook for 2021},
  pdfauthor     = {RCLL TC},
  pdfkeywords   = {RCLL, Rulebook, Challenges},
  pdfsubject    = {},
  hidelinks,
}
\usepackage{tabularx}
\usepackage{multicol}
\usepackage{multirow}
\usepackage{calc}
\usepackage{float}
\usepackage{hhline}
\usepackage{longtable}

\restylefloat{table}
%% INPUTENC %%%%%%%%%%%%%%%%%%%
\usepackage{wasysym}
\usepackage[utf8]{inputenc}
\input{macros.tex}

\newcolumntype{C}{>{\raggedright\arraybackslash}X}

\newcommand{\refsec}[1]{Section~\ref{#1}}
\newcommand{\reffig}[1]{Figure~\ref{#1}}
\newcommand{\refdef}[1]{Definition~\ref{#1}}
%\newcommand{\reflst}[1]{Listing~\ref{#1}}
\newcommand{\reflst}[1]{Figure~\ref{#1}}
\newcommand{\reftab}[1]{Table~\ref{#1}}
\usepackage{acronym}
\acrodef{BS}{base station}
\acrodef{CS}{cap station}
\acrodef{RS}{ring station}
\acrodef{SS}{storage station}
\acrodef{DS}{delivery station}
\acrodef{MPS}{modular production system}
\acrodef{RCLL}{RoboCup Logistics Leaugue}
\acrodef{TBD}{To be Discussed}

\usepackage{todonotes}
\presetkeys%
    {todonotes}%
    {inline}{}
\begin{document}
\begin{titlepage}
 \vspace*{5cm}
 \begin{center}
  \begin{LARGE}

   {\bf RoboCup Logistics League}\\[2ex]
   {\Large \bf{Challenges}}\\[4ex]
   {\Large Rules and Regulations 2021}\\[4ex]
  \end{LARGE}
  \hrule

  {\LARGE\vspace*{4ex}}
  \begin{Large}
   The Technical Committee 2012--2021\\[6ex]
  \end{Large}
  \begin{tabular}{lll}
   \textbf{Vincent Coelen}&\textbf{Christian Deppe}&\textbf{Mostafa Gomaa}\\
   \textbf{Till Hofmann}&\textbf{Ulrich Karras}&\textbf{Tim Niemueller}\\
   \textbf{Alain Rohr}&\textbf{Thomas Ulz}&\textbf{Tarik Viehmann}\\[.5em]

   Daniel Ewert&Nils Harder&S\"oren Jentzsch\\
   Nicolas Meier&Sebastian Reuter&Wataru Uemura\\
   Gerald Steinbauer&Tobias Neumann\\
  \end{tabular}
  \vfill
  Revision Date: \ac{TBD}\\
  DRAFT: Work in Progress %chktex 13
  \todo{is this up-to-date?}
 \end{center}
\end{titlepage}


\section{Introduction}
\label{sec:intro}
Our aim is to capture the main tasks from the RoboCup Logistics League in
isolated challenges, which form an additional competition in the league.
Main objectives of this new competition are:
\begin{itemize}
 \item to provide a framework that allows teams to show and evaluate their
       progress in the individual tasks of the RCLL
 \item to ease the preparation for the main competition through providing a
       simplified cost- and space-efficient setup suitable for replication in
       local labs
 \item to be attractive for both RoboCup live events and online competitions,
       where teams can participate remotely from all over the world
\end{itemize}

\section{Qualification}
\label{sec:qualification}
In addition to the regular Qualification process, teams have to participate
in a \href{https://fh-aachen.sciebo.de/s/Qkm7VfIIEIIJk1d}{RefBox workshop, see here}.
To complete the workshop, a game report dump of at least 2 Challenges
 has to be submitted to the OC.

\section{Requirements to score points}
\label{sec:scoring}
For your challenges to count as completed and score points you need to submit
the recorded video footage to the OC.
Further you need to submit your game reports from the MongoDB to the OC.
Your submission will not be counted if the footage shows human
interference with the field, machines or robots outside of
the specified rules in Section \ref{sec:operators}.

\section{Competition Area}
\subsection{Field Layout}
The competition area for the main challenges consists of a \SI{5 x 5}{\metre}
area divided in square zones of \SI{1 x 1}{\metre}. Additional challenges that
are not counting towards the scoring of the competition are carried out on
a \SI{7 x 8}{\metre} field instead (corresponding to halve the field of the
regular \ac{RCLL} field).

\todo{add picture of field with zone names}

The entire area belongs to a single team. The bottom left-most
\SI{1 x 3}{\metre} area is called the insertion zone, It does not need to have
partial walls around it and provides the starting positions for up to three
robots.

\subsection{Mockup Machines}\label{sec:machines}
In case no real \ac{MPS} stations are available, replications
(so called \emph{mockup machines}) may be used, that do not need to
physically perform the respective production steps. Instead, that work may
be carried out by a human supervisor (see \refsec{sec:operators}).
The minimum requirements for a mockup machine are specified in the following.

Mockup machines are required to have the same box-like base-shape as specified
in the RCLL rulebook \todo{Link to rulebook}.
\todo{height of the box is not specified in rulebook, only the height until
the conveyor belt}.
On top of the box a model of the conveyor belt has to be mounted, see
\todo{link to mockup model file and rulebook}.
On stations replicating a \ac{CS} a shelf as to be additionally mounted on the
front right side of the box, see
\todo{link to mockup model file and rulebook}.
On stations replicating a \ac{RS} either a shelf may be mounted as well,
or a model of the slide, which may be placed anywhere on the front right side
of the box, such that it is accessible from the front.
A model of a conveyor can be used too.
\todo{link to mockup model file and rulebook}.

The building materials for the models must be opaque, but may have any color.

In order to compete in all main challenges, a minimum number of 3 mockup
machines are required. On higher difficulty some challenges may require
4 mockup stations and some secondary challenges require 7 stations.

\subsection{Remote Setup}
In case a competition is carried out remotely, a proper local setup has to
be established and approved by the organizational committee
\todo{or is this TC responsibility?!}.
Requirements include a proper camera setup that covers the field sufficiently,
such that external viewers can verify the integrity of each challenges,
as well as an approval for every mockup machine and robot that is used.
After registration the OC will verify the camera setup in a video call.
The validation call will be scheduled individually for each team
to account for timezones.
\todo{verify correct size specs for field, robots, mockup machines}

\section{Game Play}
\subsection{Competition Scope}
\todo{X time slots of 10 minute setup time per team, followed by up to 20
	minutes of game time. Each slot can be used to solve
at most one challenge, a team can decide to fail a challenge and use another
slot to improve.
A team can use a single slot to try a challenge multiple times (with the same
field layout) once a challenge counts, it cannot be attempted again
(unless difficulty is increased).
Trying multiple times with the same field layout refers to switching
from the Production phase back to the Setup phase in the RefBox.
Restarting the RefBox will count as cancelling the challenge and you will
have to start a new run.}

All challenges (unless stated otherwise) are conducted while measuring
the execution time. The execution time is measured by the RefBox.
The fastest team in any challenge gains additional points.

\subsection{Changes compared to the Main Competition}
The tasks covered in the various challenges mostly obey the regular rules
for the \ac{RCLL}. However, some aspects are altered to simplify the setup.
The changes are not affecting existing machine communication and processing
steps, such that the challenges can be carried out on real machines as well as
on mockup versions obeying the requirements outlined in \refsec{sec:machines}.

\paragraph{Product Delivery}
The delivery procedure for finished products is altered compared to the
\ac{RCLL} rule set. In order to reduce the amount of machines required
for participation, Deliveries are made by bringing the finished product
to the insertion zone and dropping it there.

\paragraph{Ring Payments}
Easing the setup of mockup machines, it is not required to have a slide
on ring stations. Instead, a shelf may be used to place payments at the
corresponding station.

\paragraph{Ring Color Assignment}
The cost for mounting each ring color are fixed, the assignment of ring colors
is semi-fixed as teams can choose between two different options for each
challenge (\texttt{option1} or \texttt{option2} according to
\reftab{tab:ring-costs}).
\todo{make this configurable in the RefBox}

\newcommand{\colconfig}{\mathcal{RC}}
\begin{table}[!htb]
 \centering
 \begin{tabular}{l|l|l||l|l||l|l}
  & \multicolumn{4}{c||}{Ring Costs}
  & \multicolumn{2}{c}{\multirow{2}{*}{Color Assignment }}\\\cline{2-5}
  & Color  & Price & Color  & Price & \multicolumn{2}{c}{}\\\cline{2-7}
  & Yellow & 0 & Orange & 0
  & RS1: $\colconfig_1$ & RS1: $\colconfig_2$ \\
  & Green  & 1 & Blue & 2
  & RS2: $\colconfig_2$ & RS2: $\colconfig_1$ \\\hline\hline
  Configuration & \multicolumn{2}{c||}{$\colconfig_1$}
  & \multicolumn{2}{c||}{$\colconfig_2$}
  & $\texttt{option1}$ & $\texttt{option2}$\\
 \end{tabular}
 \caption{Materials}
 \label{tab:ring-costs}
\end{table}

\paragraph{Materials}\label{sec:materials}
The available material that can be used per challenge is restricted
(unless stated otherwise) per machine according to the information in
\reftab{tab:materials}.
\begin{table}[!htb]
 \centering
  \begin{tabularx}{\linewidth}{l|l}
   Machine & Available Material  \\\hline
   \ac{BS} & 2 bases of each color \\
   \ac{CS} & 3 cap-carriers (cap color choices up to each team)  \\
   \ac{RS} & 4 rings of each assigned color (8 in total)  \\
  \end{tabularx}
 \caption{Materials}
 \label{tab:materials}
\end{table}
\todo{is it actually 4 colors per ring station slot?}

\paragraph{Orders}
Unless specified otherwise, orders that have to be fulfilled in challenges
are entered through the web shop \todo{link web shop} by any member of the
competing team.

In challenges where only one \ac{RS} is present, teams are responsible to
order products which can be assembled using the available stations only.

\subsection{Field Operators}\label{sec:operators}
In challenges where mockup machines are used, the actual assembly steps have
to be performed by human supervisors. Whenever a machine is instructed,
the RefBox operator announces the required interaction. One field operator may
proceed to enter the field in order to perform the interaction. Afterwards the
field has to be left immediately.
The usual rules for replenishing resources (respecting the limited materials
\refsec{sec:materials}) apply.

\todo{maintenance rules}

\subsection{Available Challenges for the Primary Competition}
Challenges have different types and variations (difficulty levels).
The overall score of the competition is calculated by summing up the score
in the highest difficulty achieved in each of the challenge types.
The challenge types of the competition are described in
\refsec{sec:challenge-navigation}-\ref{sec:challenge-markerless}

The RefBox is used to log the scores and data for each challenge.
Once the competition is finished, $5$ bonus points are awarded each time a
team solved a challenge on a difficulty in the shortest amount of time.
\todo{implement challenges in refbox, provide tool to evaluate results}

\subsubsection{Navigation Challenge}\label{sec:challenge-navigation}
Basic navigation task with known obstacles.\\
\textbf{Task:} Drive three routes, each starting and ending in the insertion
zone while covering a given set of target positions. At each target position
the robot has to stand still (no moving or rotating) for at least 1 second. \\
Variations of this challenge depend
on the number of
available machines (see \reftab{tab:challenge-navigation}).
Multiple robots may be used to simultaneously cover different routes.
Partial points may be awarded in case only a subset of routes got covered.


\begin{table}[!htb]
 \centering
 \begin{tabular}{l|l|l}
  \multirow{2}{*}{Machines}
  & \multicolumn{2}{c}{Scoring} \\\cline{2-3}
  & first finished route & each other finished route \\\hline\hline
  2 & 10 & 2 \\
  3 & 20 & 2 \\
  4 & 25 & 2 \\
 \end{tabular}
 \caption{Navigation Challenge}
 \label{tab:challenge-navigation}
\end{table}

\subsubsection{Exploration Challenge}\label{sec:challenge-exploration}
Replicate the RCLL exploration phase.
Machine Marker detection as well as navigational skills are required to solve
this challenge.\\
\textbf{Task:} Find and report all machines on the field (type and orientation).
\\
Variable in the number of machines
(see \reftab{tab:challenge-exploration}).
\begin{table}[!htb]
 \centering
 \begin{tabular}{l|l}
  Machines & Scoring \\\hline
  2   & 10 \\
  3   & 20 \\
  4   & 30 \\
 \end{tabular}
 \caption{Exploration Challenge}
 \label{tab:challenge-exploration}
\end{table}


\subsubsection{Grasping Challenge}\label{sec:challenge-grasping}
Simple grasping task.
Each Machine has a base at output.
Robots start at cell in front of a machine output.\\
\textbf{Task:} A robot brings a base from one machine's output back to it's
input. A human supervisor places it back to the output. Repeat until all
products were placed at the respective machines input 3 times and all robots
returned to their starting positions. \\
\todo{improve description}
Variations differ by number of machines, see
\reftab{tab:challenge-grasping}. The $i$-th repetition is considered to be
successful, once all bases were placed at the respective machine input
at least $i$ times.

\begin{table}[!htb]
\centering
 \begin{tabular}{l|l|l}
  \multirow{2}{*}{Machines}
  & \multicolumn{2}{c}{Scoring} \\\cline{2-3}
  & first repetition
  & each subsequent repititon  \\\hline\hline
  1 & 10 & 2 \\
  2 & 20 & 2 \\
  3 & 25 & 2 \\
 \end{tabular}
 \caption{Grasping Challenge}
 \label{tab:challenge-grasping}
\end{table}

\subsubsection{Product Challenges}\label{sec:challenge-cx}
This section covers four types of challenges, instead of just a single one.
Each challenge corresponds to the production of a product with one of the
available complexities (C0, C1, C2, C3) in the \ac{RCLL} using either
one or two \ac{RS}.\\
For complexities C1, C2 and C3 the accumulated cost for mounting the required
rings must be equal to 1, 2 and 3, respectively.
\textbf{Task:} Produce all posted orders.\\
\begin{table}[!htb]
 \centering
 \begin{tabular}{l|l|l}
  Machines & Challenge type & Scoring \\\hline
  2 & C0 & 30 \\
  3 & C1 & 50 \\
  4 & C1 & 50 \\
  3 & C2 & 70 \\
  4 & C2 & 70 \\
  3 & C3 & 100 \\
  4 & C3 & 100 \\
 \end{tabular}
 \caption{CX Challenge}
 \label{tab:challenge-cx}
\end{table}

\subsubsection{Exploration + C0 Challenge}\label{sec:challenge-combine-exp-c0}
A simple production task on a field with unknown machine positions.
The challenge is to produce a product of complexity C0 without receiving the
machine positions at the start of the production phase, resembling a unified
exploration and production phase.\\
\textbf{Task:} Produce all posted orders.\\
Beating this challenges yields 50 points.

\subsubsection{RefBox Simulation Challenge}\label{sec:challenge-simulation}
A competition on the agent level. The RefBox provides a set of actions
that can be executed by sending dedicated commands to the RefBox via protobuf.
Hence no actual robot is required to participate.
\textbf{Task:} Play a regular RCLL game through the RefBox simulation
interface.\\
Participating in this challenge yields points based on the achieved in-game
points, see \reftab{tab:challenge-simulation}.
Additionally, the team scoring the highest points overall gets awarded another
$10$ points
\begin{table}[!htb]
 \centering
 \begin{tabular}{l|l|l|l}
  Points & Scoring \\\hline
  $[0,50)$ & 0\\% chktex 9
  $[50,150)$ & 20\\% chktex 9
  $[150,250)$ & 40 \\% chktex 9
  $[250,\infty)$ & 60 \\% chktex 9
 \end{tabular}
 \caption{CX Challenge}
 \label{tab:challenge-cx}
\end{table}


\subsubsection{Markerless Detection Challenge}\label{sec:markerless}
Image recognition challenge to classify different machine types.\\
\textbf{Task:} Autonomously label the machines shown in a set of pictures\\
As a preparation for this challenge, a data set will be supplied to
all participants which may be used for training and testing purposes.
The evaluation set for the challenge consists of a set of separate images.
\todo{pictures with multiple machines}
\begin{table}[!htb]
 \centering
 \begin{tabularx}{\linewidth}{l|l|l|l}
  \% Correctly Classified & \% Wrongly Classified & \% Not Classified
  & Scoring \\\hline
  $x$ & $y$ & $z$ & $(x-y)\cdot30$
 \end{tabularx}
 \caption{Machine Detection Challenge}
 \label{tab:challenge-markerless}
\end{table}

\subsection{Challenges for the Secondary Competition}

\subsubsection{Full game}\label{sec:challenge-full-game}
Play a full RCLL game on a field of \SI{7 x 8}{\metre} with $7$ machines
(no machines from the opposing team).
\todo{relax no-payment and delivery assumption?}
\end{document} %chktex 17
